\documentclass[11pt]{article}
\usepackage[margin=1in]{geometry}
\usepackage{setspace}
\usepackage{amsmath}
\usepackage{hyperref}

\title{Lecture Narrative: Generative Adversarial Networks}
\author{}
\date{}

\begin{document}
\maketitle
\onehalfspacing

\section*{Recording Notes}
Each subsection corresponds to one slide in the GAN lecture. These notes are written to be spoken aloud and to complement, not repeat, slide text.

\hrule
\vspace{1em}

\subsection*{Slide 1: Generative Adversarial Networks}
This lecture focuses on Generative Adversarial Networks, or GANs. GANs are a powerful framework for learning data distributions and generating realistic synthetic data, especially images.

\subsection*{Slide 2: Outline}
We will start with an introduction to generative models, then study the GAN framework and its mathematics, discuss training challenges, explore major GAN variants, and conclude with applications and research frontiers.

\subsection*{Slide 3: What Are Generative Models?}
Generative models aim to learn the underlying probability distribution of data. Once learned, they can generate new samples that resemble the training data, rather than just labeling or classifying inputs.

\subsection*{Slide 4: Generative vs Discriminative Models}
Discriminative models learn decision boundaries, such as predicting labels given inputs. Generative models instead learn how the data itself is structured, enabling synthesis, simulation, and deeper understanding.

\subsection*{Slide 5: Applications of Generative Models}
Generative models are used in image synthesis, data augmentation, anomaly detection, and style transfer. Their ability to generate data makes them valuable even when labeled data is scarce.

\subsection*{Slide 6: Types of Generative Models}
There are several families of generative models, including VAEs, normalizing flows, GANs, autoregressive models, and diffusion models. GANs are notable for producing sharp, high-quality images.

\subsection*{Slide 7: GAN Characteristics}
GANs rely on adversarial training between two networks and do not require explicit likelihood computation. Historically, they achieved state-of-the-art image quality prior to diffusion models.

\subsection*{Slide 8: The GAN Framework}
A GAN consists of a generator that maps noise to synthetic data and a discriminator that distinguishes real data from generated data. These two networks are trained simultaneously.

\subsection*{Slide 9: Adversarial Training Intuition}
The generator tries to fool the discriminator, while the discriminator tries to detect fake samples. This competition drives both networks to improve.

\subsection*{Slide 10: The GAN Minimax Game}
GAN training is formulated as a two-player minimax game. The generator minimizes the objective, while the discriminator maximizes it, leading ideally to a Nash equilibrium.

\subsection*{Slide 11: Nash Equilibrium in GANs}
At equilibrium, the generator reproduces the true data distribution, and the discriminator can no longer tell real from fake, outputting probability one-half everywhere.

\subsection*{Slide 12: Mathematical Objective}
The original GAN objective combines expectations over real data and generated data. This formulation provides a theoretical foundation but introduces practical optimization challenges.

\subsection*{Slide 13: Optimal Discriminator}
For a fixed generator, the optimal discriminator has a closed-form solution depending on the ratio of real and generated data distributions.

\subsection*{Slide 14: Training Algorithm}
GANs are trained by alternating updates: multiple discriminator updates followed by a generator update. In practice, generator loss is modified to avoid vanishing gradients.

\subsection*{Slide 15: Convergence and Global Optimality}
At the global optimum, the GAN objective reduces to minimizing the Jensen--Shannon divergence between real and generated distributions.

\subsection*{Slide 16: Training Instability Issues}
GAN training is notoriously unstable. Common issues include vanishing gradients, mode collapse, and non-convergence.

\subsection*{Slide 17: Mode Collapse}
In mode collapse, the generator produces limited diversity, covering only a few modes of the data distribution while ignoring others.

\subsection*{Slide 18: Wasserstein GAN}
WGAN replaces the JS divergence with the Wasserstein distance, providing smoother gradients and more meaningful loss values.

\subsection*{Slide 19: WGAN with Gradient Penalty}
WGAN-GP improves stability by enforcing the Lipschitz constraint using a gradient penalty rather than weight clipping.

\subsection*{Slide 20: DCGAN}
DCGAN introduced architectural guidelines that enabled stable GAN training using convolutional neural networks and learned meaningful latent representations.

\subsection*{Slide 21: Conditional GANs}
Conditional GANs incorporate auxiliary information such as class labels, allowing controlled generation and improved sample quality.

\subsection*{Slide 22: StyleGAN Evolution}
The StyleGAN series introduced style-based generators, improved training stability, and high-quality, controllable image synthesis.

\subsection*{Slide 23: BigGAN}
BigGAN demonstrated that scaling model size, batch size, and data can dramatically improve image fidelity at high resolutions.

\subsection*{Slide 24: Text-to-Image Models}
Modern systems like DALL·E and Imagen combine generative modeling with powerful language representations, enabling semantic image synthesis.

\subsection*{Slide 25: Stable Diffusion}
Stable Diffusion operates in latent space, making text-to-image generation computationally efficient while maintaining high quality.

\subsection*{Slide 26: Medical Applications}
GANs are used in medical imaging for data augmentation and anomaly detection, but require careful validation and ethical consideration.

\subsection*{Slide 27: GANs vs Diffusion Models}
Diffusion models offer more stable training and better mode coverage, while GANs excel in fast sampling. Hybrid approaches aim to combine their strengths.

\subsection*{Slide 28: Conclusion and Outlook}
GANs revolutionized generative modeling and remain an active research area. Future work includes 3D generation, video synthesis, multimodal learning, and responsible deployment.

\end{document}
