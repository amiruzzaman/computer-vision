\documentclass{beamer}

\usetheme{Madrid}
\usecolortheme{default}

\usepackage{amsmath}
\usepackage{amsfonts}
\usepackage{graphicx}
\usepackage{bm}

\title{Introduction to Computer Vision}
\subtitle{Image Processing Fundamentals}
\author{Dr.\ Amir}
\institute{Department of Computer Science}
%\date{\today}
\date{}

\begin{document}

%------------------------------------------------
\begin{frame}
\titlepage
\end{frame}

%------------------------------------------------
\begin{frame}{Lecture Outline}
\tableofcontents
\end{frame}

%------------------------------------------------
\section{What Is Computer Vision}

\begin{frame}{Computer Vision}
\begin{itemize}
    \item Computer Vision enables machines to \textbf{see, analyze, and understand images}
    \item Core tasks:
    \begin{itemize}
        \item Image processing
        \item Feature extraction
        \item Object detection
        \item Image classification
    \end{itemize}
\end{itemize}
\end{frame}

%------------------------------------------------
\section{Image as a Numerical Signal}

\begin{frame}{Image as a Matrix}
\begin{itemize}
    \item A grayscale image is a 2D matrix:
\end{itemize}

\[
I(x,y) \in [0, 255]
\]

\[
\bm{I} =
\begin{bmatrix}
52 & 55 & 61 \\
63 & 59 & 55 \\
70 & 61 & 64
\end{bmatrix}
\]

\begin{itemize}
    \item Each value represents pixel intensity
    \item Color images use 3 channels (RGB)
\end{itemize}
\end{frame}

%------------------------------------------------
\begin{frame}{Color Images}
\[
\bm{I}(x,y) = \begin{bmatrix}
R(x,y) \\
G(x,y) \\
B(x,y)
\end{bmatrix}
\]

\begin{itemize}
    \item Each channel is a matrix
    \item Stored as a 3D array: height × width × 3
\end{itemize}
\end{frame}

%------------------------------------------------
\section{Image Filtering}

\begin{frame}{Image Filtering}
\begin{itemize}
    \item Filtering modifies pixel values using neighbors
    \item Used for:
    \begin{itemize}
        \item Noise reduction
        \item Smoothing
        \item Edge enhancement
    \end{itemize}
\end{itemize}
\end{frame}

%------------------------------------------------
\begin{frame}{Convolution Operation}
\[
I'(x,y) = \sum_{i=-k}^{k} \sum_{j=-k}^{k} I(x+i, y+j)\,K(i,j)
\]

\begin{itemize}
    \item $K$ is a kernel (filter)
    \item Applied to every pixel
\end{itemize}
\end{frame}

%------------------------------------------------
\begin{frame}{Common Filters}
\textbf{Mean Filter}
\[
K = \frac{1}{9}
\begin{bmatrix}
1 & 1 & 1 \\
1 & 1 & 1 \\
1 & 1 & 1
\end{bmatrix}
\]

\vspace{0.3cm}

\textbf{Gaussian Filter}
\[
G(x,y) = \frac{1}{2\pi\sigma^2}
e^{-\frac{x^2+y^2}{2\sigma^2}}
\]
\end{frame}

%------------------------------------------------
\section{Edge Detection}

\begin{frame}{Edges in Images}
\begin{itemize}
    \item Edges correspond to \textbf{intensity changes}
    \item Detected using image gradients
\end{itemize}
\end{frame}

%------------------------------------------------
\begin{frame}{Image Gradient}
\[
\nabla I =
\begin{bmatrix}
\frac{\partial I}{\partial x} \\
\frac{\partial I}{\partial y}
\end{bmatrix}
\]

\[
|\nabla I| = \sqrt{G_x^2 + G_y^2}
\]

\begin{itemize}
    \item High gradient magnitude $\Rightarrow$ edge
\end{itemize}
\end{frame}

%------------------------------------------------
\begin{frame}{Sobel Operator}
\[
G_x =
\begin{bmatrix}
-1 & 0 & 1 \\
-2 & 0 & 2 \\
-1 & 0 & 1
\end{bmatrix}
\quad
G_y =
\begin{bmatrix}
-1 & -2 & -1 \\
0 & 0 & 0 \\
1 & 2 & 1
\end{bmatrix}
\]

\begin{itemize}
    \item Approximates image derivatives
\end{itemize}
\end{frame}

%------------------------------------------------
\section{Image Classification}

\begin{frame}{What Is Image Classification?}
\begin{itemize}
    \item Assigning a label to an image
    \item Examples:
    \begin{itemize}
        \item Cat vs Dog
        \item Tumor vs Normal
        \item Road sign recognition
    \end{itemize}
\end{itemize}
\end{frame}

%------------------------------------------------
\begin{frame}{Image as Feature Vector}
\begin{itemize}
    \item Image is flattened into a vector:
\end{itemize}

\[
\bm{x} = [p_1, p_2, \dots, p_n]
\]

\begin{itemize}
    \item Classifier learns mapping:
\end{itemize}

\[
f(\bm{x}) = y
\]
\end{frame}

%------------------------------------------------
\begin{frame}{Classification Pipeline}
\begin{enumerate}
    \item Image acquisition
    \item Preprocessing
    \item Feature extraction
    \item Classification model
\end{enumerate}
\end{frame}

%------------------------------------------------
\begin{frame}{Summary}
\begin{itemize}
    \item Images are numerical matrices
    \item Filtering modifies local neighborhoods
    \item Edges are detected via gradients
    \item Classification assigns semantic meaning
\end{itemize}
\end{frame}

\end{document}
