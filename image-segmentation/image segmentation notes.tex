\documentclass[11pt]{article}
\usepackage[margin=1in]{geometry}
\usepackage{setspace}
\usepackage{amsmath}
\usepackage{hyperref}

\title{Lecture Narrative: Image Segmentation}
\author{}
\date{}

\begin{document}
\maketitle
\onehalfspacing

\section*{Recording Notes}
Each subsection corresponds to one slide in the Image Segmentation lecture. Slide numbers follow the order of the presentation from title to the final clustering discussion.

\hrule
\vspace{1em}

\subsection*{Slide 1: Image Segmentation}
This lecture introduces image segmentation, which is the task of partitioning an image into meaningful regions. Segmentation is a foundational problem in computer vision and often serves as the first step toward higher-level understanding.

\subsection*{Slide 2: Source}
This material is adapted from lecture notes available at Imperial College London. The focus here is on classical segmentation methods rather than modern deep learning approaches.

\subsection*{Slide 3: Introduction to Image Segmentation}
The goal of image segmentation is to divide an image into regions that are meaningful for a particular application. The definition of “meaningful” depends heavily on the task we are trying to solve.

\subsection*{Slide 4: Measurements Used for Segmentation}
Segmentation can be based on many different image measurements, including greylevel intensity, color, texture, depth, or motion. Different applications require different cues.

\subsection*{Slide 5: Segmentation as an Initial Step}
Segmentation is usually an early and crucial step in a larger vision pipeline. Errors at this stage can propagate and negatively affect later processing.

\subsection*{Slide 6: Applications of Segmentation}
Applications include identifying objects for measurement, segmenting moving objects for video compression, and separating objects by depth for robotic navigation.

\subsection*{Slide 7: Greyscale-Based Segmentation Example}
This example shows segmentation based purely on greyscale intensity. Using a very simple intensity model often leads to incorrect object labeling.

\subsection*{Slide 8: Limitations of Greyscale Segmentation}
Greyscale-based segmentation struggles when object and background intensities overlap, resulting in ambiguous boundaries.

\subsection*{Slide 9: Texture-Based Segmentation Example}
Segmentation based on texture can distinguish regions even when greyscale values vary within objects.

\subsection*{Slide 10: Motivation for Texture Segmentation}
Texture allows us to capture local patterns rather than raw intensity, making segmentation more robust in complex scenes.

\subsection*{Slide 11: Motion-Based Segmentation}
Segmentation based on motion requires estimating optical flow, which introduces uncertainty. The segmentation relies on estimated motion rather than true motion.

\subsection*{Slide 12: Challenges in Motion Segmentation}
Errors in optical flow estimation directly affect segmentation accuracy, making this a challenging problem.

\subsection*{Slide 13: Depth-Based Segmentation}
Depth segmentation uses distance measurements, often from sensors like laser range finders, to separate objects by distance.

\subsection*{Slide 14: Robotics Motivation}
Depth-based segmentation is especially useful in robotics, where understanding object distance enables navigation and path planning.

\subsection*{Slide 15: Example Images}
This slide shows an original image, its corresponding range image, and the resulting segmented image.

\subsection*{Slide 16: Histogram-Based Segmentation}
We now focus on simple segmentation techniques based on the greylevel histogram of an image, specifically thresholding and clustering.

\subsection*{Slide 17: Test Images}
We will examine noise-free, low-noise, and high-noise images to understand how noise affects segmentation performance.

\subsection*{Slide 18: Histogram Interpretation}
Histograms allow us to visualize the distribution of greylevels and assess how well object and background are separated.

\subsection*{Slide 19: Noise-Free Histogram}
In the noise-free case, the histogram consists of two spikes corresponding to object and background intensities.

\subsection*{Slide 20: Low-Noise Histogram}
With low noise, the spikes become broader peaks, but the object and background are still distinguishable.

\subsection*{Slide 21: High-Noise Histogram}
With high noise, the peaks merge into a single distribution, making segmentation much harder.

\subsection*{Slide 22: Signal-to-Noise Ratio}
We define the signal-to-noise ratio in terms of the mean greylevels of object and background and the noise standard deviation.

\subsection*{Slide 23: SNR for Test Images}
For the noise-free image, the SNR is infinite. For the low-noise image, it is approximately 5, and for the high-noise image, it drops to around 2.

\subsection*{Slide 24: Greylevel Thresholding}
Thresholding is one of the simplest segmentation methods. A single threshold value separates object from background.

\subsection*{Slide 25: Histogram Valley}
In the low-noise case, there is a clear valley between the object and background peaks where a threshold can be placed.

\subsection*{Slide 26: Threshold Definition}
The thresholding rule assigns pixels to object or background depending on whether their greylevel is below or above the threshold.

\subsection*{Slide 27: Choosing the Threshold}
The key challenge is selecting an appropriate threshold value. Several strategies exist, including interactive and adaptive methods.

\subsection*{Slide 28: Minimization Approach}
We focus on a minimization approach that selects the threshold by minimizing the within-group variance.

\subsection*{Slide 29: Idealized Histogram}
An ideal object-background histogram helps illustrate how thresholding divides pixels into two groups.

\subsection*{Slide 30: Within-Group Variance}
Each threshold defines two groups, each with its own mean and variance. The goal is to make each group as homogeneous as possible.

\subsection*{Slide 31: Optimal Threshold}
The optimal threshold minimizes the weighted sum of variances within the object and background groups.

\subsection*{Slide 32: Group Definitions}
Object pixels have greylevels less than or equal to the threshold, while background pixels exceed it.

\subsection*{Slide 33: Prior Probabilities}
The prior probabilities of object and background are computed directly from the histogram.

\subsection*{Slide 34: Group Statistics}
The mean and variance of each group can be derived from the histogram and threshold.

\subsection*{Slide 35: Optimization}
The within-group variance is evaluated for all possible thresholds, requiring only 256 comparisons for an 8-bit image.

\subsection*{Slide 36: Threshold Result}
For the low-noise image, the optimal threshold is approximately 124, roughly midway between the two peaks.

\subsection*{Slide 37: Applying the Threshold}
Applying this threshold segments both the low-noise and high-noise images.

\subsection*{Slide 38: Low-Noise Segmentation Result}
The segmentation works reasonably well for the low-noise image.

\subsection*{Slide 39: High-Noise Segmentation Result}
In the high-noise case, significant pixel misclassification is visible.

\subsection*{Slide 40: Misclassification Analysis}
Misclassification arises from overlap between object and background histograms.

\subsection*{Slide 41: Fundamental Limitation}
No threshold can perfectly separate object and background when their greylevel distributions overlap.

\subsection*{Slide 42: Error Probability}
The probability of error increases as histogram overlap increases.

\subsection*{Slide 43: Introduction to Greylevel Clustering}
We now consider greylevel clustering as an alternative to thresholding.

\subsection*{Slide 44: Clustering Idea}
Clustering separates greylevels into groups based on proximity to cluster centers.

\subsection*{Slide 45: Cluster Centers}
Two cluster centers represent object and background intensities.

\subsection*{Slide 46: Nearest Neighbor Assignment}
Each greylevel is assigned to the cluster whose center is closest.

\subsection*{Slide 47: Relation to K-Means}
This approach is a simple case of K-means clustering, which is widely used in practice.

\subsection*{Slide 48: Partitioning the Data}
The greylevels are partitioned into two sets corresponding to object and background.

\subsection*{Slide 49: Updating Cluster Means}
Cluster centers are updated as the mean greylevel of assigned pixels.

\subsection*{Slide 50: Chicken-and-Egg Problem}
Cluster assignments depend on cluster means, and cluster means depend on assignments.

\subsection*{Slide 51: Iterative Solution}
This dependency is resolved using an iterative algorithm.

\subsection*{Slide 52: Iterative Algorithm Steps}
The algorithm alternates between updating cluster means and reassigning pixels.

\subsection*{Slide 53: Convergence Questions}
Two key questions arise: does the algorithm converge, and what does it converge to?

\subsection*{Slide 54: Cost Function}
We define a cost function measuring within-cluster variance.

\subsection*{Slide 55: Updating the Cost}
Each iteration reduces or maintains the cost function value.

\subsection*{Slide 56: Proof of Convergence}
Because the cost is non-increasing and bounded below, the algorithm must converge.

\subsection*{Slide 57: Interpretation of the Result}
The algorithm converges to a local minimum of within-cluster variance.

\subsection*{Slide 58: Relation to Thresholding}
For well-separated clusters, clustering and thresholding yield similar results.

\subsection*{Slide 59: Performance Comparison}
Both methods degrade as object and background distributions overlap.

\subsection*{Slide 60: Key Takeaways}
Simple histogram-based methods are intuitive and efficient but fundamentally limited by noise and overlap.

\subsection*{Slide 61: Transition Forward}
These limitations motivate more advanced segmentation methods based on spatial context, texture, and learning-based models.

\end{document}
