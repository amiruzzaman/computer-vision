\documentclass{beamer}

% Theme and packages
\usetheme{Madrid}
\usepackage{amsmath, amssymb}
\usepackage{graphicx}
\usepackage{booktabs}

% Title information
\title[Camera Geometry]{Intrinsic and Extrinsic Parameters\\Rigid Transformations, Homogeneous Coordinates,\\and Perspective Projection}
\author{Amir}
\institute{}
\date{}

\begin{document}

%------------------------------------------------
\begin{frame}
  \titlepage
\end{frame}

%------------------------------------------------
\begin{frame}{Lecture Outline}
\begin{itemize}
  \item Camera models and coordinate systems
  \item Rigid transformations in 3D
  \item Homogeneous coordinates
  \item Intrinsic and extrinsic camera parameters
  \item Perspective projection model
  \item Examples and discussion questions
\end{itemize}
\end{frame}

%------------------------------------------------
\section{Camera Geometry Basics}

\begin{frame}{Why Camera Geometry?}
\begin{itemize}
  \item Relates 3D world points to 2D image measurements
  \item Foundation of computer vision tasks:
  \begin{itemize}
    \item Camera calibration
    \item 3D reconstruction
    \item Visual odometry and SLAM
  \end{itemize}
  \item Separates \emph{geometry} from \emph{appearance}
\end{itemize}
\end{frame}

%------------------------------------------------
\begin{frame}{Coordinate Systems}
\begin{itemize}
  \item \textbf{World coordinate system}: fixed reference in the scene
  \item \textbf{Camera coordinate system}: origin at the camera center
  \item \textbf{Image coordinate system}: pixel coordinates on the image plane
\end{itemize}

\vspace{0.3cm}
A key goal: map a 3D point
\[
\mathbf{X}_w = (X, Y, Z)^T
\]
from world coordinates to an image point
\[
\mathbf{x} = (u, v)^T.
\]
\end{frame}

%------------------------------------------------
\section{Rigid Transformations}

\begin{frame}{Rigid Transformations in 3D}
\textbf{Definition:} A rigid transformation preserves distances and angles.

\vspace{0.3cm}
It consists of:
\begin{itemize}
  \item Rotation $\mathbf{R} \in SO(3)$
  \item Translation $\mathbf{t} \in \mathbb{R}^3$
\end{itemize}

\vspace{0.3cm}
Mapping from world to camera coordinates:
\[
\mathbf{X}_c = \mathbf{R}\,\mathbf{X}_w + \mathbf{t}
\]
\end{frame}

%------------------------------------------------
\begin{frame}{Rotation Matrix Properties}
A rotation matrix $\mathbf{R}$ satisfies:
\begin{itemize}
  \item $\mathbf{R}^T \mathbf{R} = \mathbf{I}$ (orthonormality)
  \item $\det(\mathbf{R}) = 1$
\end{itemize}

\vspace{0.3cm}
\textbf{Interpretation:}
\begin{itemize}
  \item Rows (or columns) represent camera axes
  \item Encodes camera orientation in space
\end{itemize}
\end{frame}

%------------------------------------------------
\section{Homogeneous Coordinates}

\begin{frame}{Homogeneous Coordinates}
\textbf{Motivation:}
\begin{itemize}
  \item Allow translation to be represented as matrix multiplication
  \item Enable projective transformations
\end{itemize}

\vspace{0.3cm}
A 3D point in homogeneous coordinates:
\[
\tilde{\mathbf{X}} = (X, Y, Z, 1)^T
\]

A 2D image point:
\[
\tilde{\mathbf{x}} = (u, v, 1)^T
\]
\end{frame}

%------------------------------------------------
\begin{frame}{Rigid Transformation in Homogeneous Form}
Using homogeneous coordinates:
\[
\tilde{\mathbf{X}}_c =
\begin{bmatrix}
\mathbf{R} & \mathbf{t} \\
\mathbf{0}^T & 1
\end{bmatrix}
\tilde{\mathbf{X}}_w
\]

\vspace{0.3cm}
This $4 \times 4$ matrix compactly represents rotation and translation.
\end{frame}

%------------------------------------------------
\section{Intrinsic and Extrinsic Parameters}

\begin{frame}{Extrinsic Parameters}
\textbf{Definition:}
Extrinsic parameters describe the \emph{pose} of the camera in the world.

\vspace{0.3cm}
They consist of:
\begin{itemize}
  \item Rotation $\mathbf{R}$
  \item Translation $\mathbf{t}$
\end{itemize}

\vspace{0.3cm}
They answer the question:
\begin{quote}
\emph{Where is the camera, and how is it oriented?}
\end{quote}
\end{frame}

%------------------------------------------------
\begin{frame}{Intrinsic Parameters}
\textbf{Definition:}
Intrinsic parameters describe the internal geometry of the camera.

\vspace{0.3cm}
They are encoded in the intrinsic matrix:
\[
\mathbf{K} =
\begin{bmatrix}
 f_x & s & c_x \\
 0   & f_y & c_y \\
 0   & 0   & 1
\end{bmatrix}
\]

\begin{itemize}
  \item $(f_x, f_y)$: focal lengths in pixel units
  \item $(c_x, c_y)$: principal point
  \item $s$: skew (often zero)
\end{itemize}
\end{frame}

%------------------------------------------------
\section{Perspective Projection}

\begin{frame}{Pinhole Camera Model}
\textbf{Assumption:}
\begin{itemize}
  \item All light rays pass through a single point (camera center)
\end{itemize}

\vspace{0.3cm}
Perspective projection from camera coordinates:
\[
(x, y) = \left( \frac{X_c}{Z_c}, \frac{Y_c}{Z_c} \right)
\]

\vspace{0.3cm}
This explains:
\begin{itemize}
  \item Scale change with depth
  \item Vanishing points
\end{itemize}
\end{frame}

%------------------------------------------------
\begin{frame}{Full Camera Projection Equation}
Combining extrinsic and intrinsic parameters:
\[
\tilde{\mathbf{x}} \sim \mathbf{K}\,[\mathbf{R} \;|\; \mathbf{t}]\,\tilde{\mathbf{X}}_w
\]

\vspace{0.3cm}
Where:
\begin{itemize}
  \item $\sim$ denotes equality up to scale
  \item $\mathbf{P} = \mathbf{K}[\mathbf{R}|\mathbf{t}]$ is the \emph{camera projection matrix}
\end{itemize}
\end{frame}

%------------------------------------------------
\section{Examples}

\begin{frame}{Example: Simple Projection}
Assume:
\begin{itemize}
  \item $\mathbf{R} = \mathbf{I}$, $\mathbf{t} = \mathbf{0}$
  \item $f_x = f_y = f$, $c_x = c_y = 0$
\end{itemize}

Then:
\[
\mathbf{x} = \left( f \frac{X}{Z},\; f \frac{Y}{Z} \right)
\]

\vspace{0.3cm}
\textbf{Interpretation:}
Objects farther from the camera appear smaller.
\end{frame}

%------------------------------------------------
\section{Discussion Questions}

\begin{frame}{Conceptual Questions}
\begin{enumerate}
  \item Why are intrinsic parameters independent of camera pose?
  \item What happens to the projection when $Z \rightarrow 0$?
  \item Why are homogeneous coordinates essential for perspective projection?
  \item How would lens distortion violate the pinhole model assumptions?
\end{enumerate}
\end{frame}

%------------------------------------------------
\section{References}

\begin{frame}{References}
\footnotesize
\begin{thebibliography}{9}

\bibitem{HartleyZisserman}
R. Hartley and A. Zisserman,
\emph{Multiple View Geometry in Computer Vision},
2nd ed., Cambridge University Press, 2004.

\bibitem{Szeliski}
R. Szeliski,
\emph{Computer Vision: Algorithms and Applications},
Springer, 2011.

\bibitem{Faugeras}
O. Faugeras,
\emph{Three-Dimensional Computer Vision},
MIT Press, 1993.

\end{thebibliography}
\end{frame}

\end{document}
