\documentclass{beamer}

% =====================================================
% THEME & PACKAGES
% =====================================================
\usetheme{Madrid}
\usecolortheme{default}

\usepackage{amsmath, amssymb}
\usepackage{graphicx}
\usepackage{tikz}
\usepackage{booktabs}

% =====================================================
% TITLE INFORMATION (AS REQUESTED)
% =====================================================
\title[Camera Models for Computer Vision]{Mathematical Foundations of Camera Models for Computer Vision}
\author{Amir}
\institute{Department of Computer Science}
\date{} % explicitly no date

\begin{document}

% =====================================================
% TITLE SLIDE
% =====================================================
\begin{frame}
  \titlepage
\end{frame}

% =====================================================
% LEARNING PHILOSOPHY
% =====================================================
\begin{frame}{How to Read These Slides}
These slides are designed as a \textbf{guided tutorial}, not a reference sheet.

\begin{itemize}
  \item No prior linear algebra is assumed
  \item Every symbol will be explained
  \item Every equation will solve a concrete camera problem
\end{itemize}
\end{frame}

\begin{frame}{Why This Much Detail?}
Camera models sit at the foundation of computer vision.

\begin{itemize}
  \item If the camera model is unclear, algorithms appear magical
  \item If the camera model is clear, algorithms become logical
\end{itemize}

\medskip
\textbf{Our goal is understanding, not memorization.}
\end{frame}

% =====================================================
% WHY CAMERAS MATTER FOR CV
% =====================================================
\begin{frame}{What Is Computer Vision?}
Computer vision studies how machines:
\begin{itemize}
  \item Receive visual input
  \item Interpret images and videos
  \item Make decisions based on visual data
\end{itemize}
\end{frame}

\begin{frame}{Where Do Images Come From?}
\begin{itemize}
  \item Images are captured by cameras
  \item Cameras convert light into numbers
  \item These numbers depend on geometry
\end{itemize}
\end{frame}

\begin{frame}{Why Camera Understanding Is Essential}
Without understanding the camera:
\begin{itemize}
  \item Depth is ambiguous
  \item Scale is misleading
  \item Motion cannot be inferred reliably
\end{itemize}
\end{frame}

\begin{frame}{A Core Problem in Vision}
\begin{center}
\textbf{The world is 3D, but images are 2D.}
\end{center}

This mismatch is the root of many vision challenges.
\end{frame}

% =====================================================
% WHY MATHEMATICS IS NEEDED
% =====================================================
\begin{frame}{Why Mathematics Is Unavoidable}
Cameras follow:
\begin{itemize}
  \item Physical laws (optics)
  \item Geometric rules (projection)
\end{itemize}

\medskip
Mathematics allows us to express these rules precisely.
\end{frame}

\begin{frame}{What Goes Wrong Without Math}
\begin{itemize}
  \item Same image $\rightarrow$ different possible scenes
  \item Algorithms fail without explanation
  \item No way to reason about errors
\end{itemize}
\end{frame}

\begin{frame}{What Mathematics Gives Us}
\begin{itemize}
  \item Predictability
  \item Repeatability
  \item Interpretability
\end{itemize}
\end{frame}

% =====================================================
% SCALARS
% =====================================================
\begin{frame}{Scalars: The Simplest Mathematical Objects}
A \textbf{scalar} is a single number.

Examples:
\begin{itemize}
  \item Pixel intensity
  \item Distance
  \item Time
\end{itemize}
\end{frame}

\begin{frame}{Why Scalars Are Not Enough}
A camera measurement includes:
\begin{itemize}
  \item Horizontal position
  \item Vertical position
  \item Depth
\end{itemize}

\medskip
One number cannot describe all of this.
\end{frame}

% =====================================================
% COORDINATES
% =====================================================
\begin{frame}{Coordinates: Describing Position}
Coordinates answer one question:

\begin{center}
\textbf{Where is this point?}
\end{center}

Example in 2D:
\[
(x, y)
\]
\end{frame}

\begin{frame}{2D Coordinates and Images}
\begin{itemize}
  \item Images live on a 2D grid
  \item Each pixel has an $(x, y)$ location
\end{itemize}
\end{frame}

\begin{frame}{3D Coordinates and the Physical World}
\begin{itemize}
  \item The real world has depth
  \item Depth is represented by $z$
\end{itemize}

\[
(x, y, z)
\]
\end{frame}

% =====================================================
% VECTORS
% =====================================================
\begin{frame}{Vectors: Grouping Numbers Together}
A \textbf{vector} is an ordered collection of numbers.

\[
\mathbf{x} =
\begin{bmatrix}
x \\
y
\end{bmatrix}
\]
\end{frame}

\begin{frame}{Why Vectors Matter for Cameras}
In camera modeling:
\begin{itemize}
  \item A point is a vector
  \item A direction is a vector
  \item A ray is a vector
\end{itemize}
\end{frame}

\begin{frame}{Vector Addition: Motion}
\[
\begin{bmatrix}
1 \\ 2
\end{bmatrix}
+
\begin{bmatrix}
2 \\ 1
\end{bmatrix}
=
\begin{bmatrix}
3 \\ 3
\end{bmatrix}
\]

Used to represent translation.
\end{frame}
