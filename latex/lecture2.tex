\documentclass{beamer}
\mode<presentation>{
  \usetheme{Madrid}
  \setbeamertemplate{navigation symbols}{}
}
\usepackage{amsmath,amssymb}
\usepackage{graphicx}
\usepackage{caption}
\usepackage{booktabs}
\usepackage{hyperref}

% Title info
\title[Image Formation]{Image Formation \& Geometric Camera Models}
\author{Adapted from: Computer Vision Lecture 2.2 (video) \\ and Geometric Image Formation notes}
\institute{Prepared by Your Name}
\date{\today}

\begin{document}

% Title slide
\begin{frame}
  \titlepage
\end{frame}

% Outline
\begin{frame}{Outline}
  \tableofcontents
\end{frame}

\section{Learning Objectives}
\begin{frame}{Learning Objectives}
  \begin{itemize}
    \item Understand the pinhole camera model and perspective projection.
    \item Represent points and transformations using homogeneous coordinates.
    \item Build the camera projection matrix and separate intrinsics/extrinsics.
    \item Recognize common lens distortions and basics of calibration.
  \end{itemize}
\end{frame}

\section{Why geometric image formation?}
\begin{frame}{Why geometric image formation?}
  \begin{itemize}
    \item Relates 3D scene geometry to 2D images.
    \item Foundation for stereo, structure-from-motion, pose estimation, calibration.
    \item Mathematical models let us invert or reason about scene geometry.
  \end{itemize}
\end{frame}

\section{Pinhole Camera Model}
\begin{frame}{Pinhole camera model}
  \begin{columns}
    \column{0.55\textwidth}
      \textbf{Intuition:} A pinhole camera projects 3D points onto an image plane through a single center of projection.
      \vspace{6pt}
      \begin{align*}
        \text{3D point } &\mathbf{X} = (X,Y,Z)^\top \\
        \text{Image point } & (x,y) = \left(\frac{f X}{Z}, \frac{f Y}{Z}\right)
      \end{align*}
      where \(f\) is the focal length.
    \column{0.45\textwidth}
      % Placeholder for optional figure
      \begin{figure}
        \centering
        \includegraphics[width=0.9\linewidth]{pinhole_diagram_placeholder.png}
        \caption*{Pinhole geometry (replace with own figure).}
      \end{figure}
  \end{columns}
\end{frame}

\section{Homogeneous Coordinates}
\begin{frame}{Homogeneous coordinates}
  \begin{itemize}
    \item Represent Euclidean 2D/3D points with extra coordinate for linear transformations.
    \item 3D point: \(\tilde{\mathbf{X}} = (X, Y, Z, 1)^\top\).
    \item 2D image (homogeneous): \(\tilde{\mathbf{x}} = (x, y, 1)^\top\).
    \item Perspective projection becomes linear in homogeneous coordinates:
    \[
      s \tilde{\mathbf{x}} = \mathbf{P} \, \tilde{\mathbf{X}}
    \]
    where \(s\) is scale and \(\mathbf{P}\) is the \(3\times4\) projection matrix.
  \end{itemize}
\end{frame}

\section{Projection Matrix}
\begin{frame}{Camera projection matrix}
  \begin{itemize}
    \item Full camera matrix: \(\mathbf{P} = \mathbf{K} [\mathbf{R} \mid \mathbf{t}]\)
    \item \(\mathbf{K}\) (intrinsics) and \([\mathbf{R}\mid\mathbf{t}]\) (extrinsics):
    \[
      \mathbf{K} =
      \begin{bmatrix}
        f_x & s & c_x \\
        0 & f_y & c_y \\
        0 & 0 & 1
      \end{bmatrix}
    \]
    where \(f_x,f_y\) are focal lengths (pixels), \(s\) is skew, and \(c_x,c_y\) is principal point.
  \end{itemize}
\end{frame}

\begin{frame}{Interpretation of extrinsics}
  \begin{itemize}
    \item World-to-camera: \(\mathbf{X}_c = \mathbf{R}\mathbf{X}_w + \mathbf{t}\).
    \item \(\mathbf{R}\) is a 3×3 rotation, \(\mathbf{t}\) is translation — together they place the camera in the world.
    \item Combined into projection: \(\tilde{\mathbf{x}} \sim \mathbf{K}[\mathbf{R} \mid \mathbf{t}] \tilde{\mathbf{X}}_w\).
  \end{itemize}
\end{frame}

\section{Common Camera Models}
\begin{frame}{Other camera approximations}
  \begin{itemize}
    \item \textbf{Orthographic / scaled orthographic:} approximates perspective when depth variation is small.
    \item \textbf{Affine camera:} linear mapping, drops division by \(Z\).
    \item \textbf{Omnidirectional / generalized cameras:} spherical, catadioptric or multi-view sensors.
  \end{itemize}
\end{frame}

\section{Lens Distortion}
\begin{frame}{Lens distortion (radial \& tangential)}
  \begin{itemize}
    \item Real lenses introduce radial distortion \(r\) (barrel/pincushion) and tangential terms due to misalignment.
    \item Simple radial correction (2–3 coefficients):
    \[
      x_{\text{dist}} = x(1 + k_1 r^2 + k_2 r^4 + \dots)
    \]
    where \(r^2 = x^2 + y^2\) in normalized image coordinates.
  \end{itemize}
\end{frame}

\section{Camera Calibration}
\begin{frame}{Camera calibration}
  \begin{itemize}
    \item Estimate \(\mathbf{K}\), \(\mathbf{R}\), \(\mathbf{t}\) (and distortion) from known 3D–2D correspondences.
    \item Common toolboxes: Bouguet's Camera Calibration Toolbox, OpenCV calibrateCamera.
    \item Calibration yields parameters used to undistort images and recover metric geometry.
  \end{itemize}
\end{frame}

\section{Worked Example (Projection)}
\begin{frame}{Example: projecting a 3D point}
  Given \(\mathbf{X}=(X,Y,Z)^\top\), compute image point:
  \[
    s
    \begin{bmatrix} u \\ v \\ 1 \end{bmatrix}
    =
    \begin{bmatrix}
      f_x & s & c_x \\
      0 & f_y & c_y \\
      0 & 0 & 1
    \end{bmatrix}
    \big[\mathbf{R}\mid\mathbf{t}\big]
    \begin{bmatrix} X \\ Y \\ Z \\ 1 \end{bmatrix}
  \]
  Solve for \(u = \frac{s_u}{s_w},\; v=\frac{s_v}{s_w}\).
\end{frame}

\section{Takeaways}
\begin{frame}{Takeaways}
  \begin{itemize}
    \item The pinhole model and homogeneous coordinates give a compact, linear framework for projection.
    \item Intrinsics vs extrinsics: intrinsics describe the camera sensor/lens, extrinsics describe camera pose.
    \item Calibration and distortion modeling are essential for accurate geometric tasks.
  \end{itemize}
\end{frame}

\section{References}
\begin{frame}{References}
  \begin{itemize}
    \item Lecture video: ``Computer Vision - Lecture 2.2 (Image Formation: Geometric ...)'' (YouTube). % cite in chat
    \item Lecture notes: Geometric Image Formation (example course notes). % cite in chat
    \item R. Szeliski, \textit{Computer Vision: Algorithms and Applications} and Hartley \& Zisserman, \textit{Multiple View Geometry}.
  \end{itemize}
\end{frame}

\begin{frame}{Appendix: Useful formulas}
  \small
  \begin{itemize}
    \item \(\mathbf{P} = \mathbf{K}[\mathbf{R}\mid \mathbf{t}]\)
    \item Projection (homogeneous): \( s\tilde{\mathbf{x}} = \mathbf{P}\tilde{\mathbf{X}} \)
    \item Intrinsics matrix \(\mathbf{K}\) shown earlier.
  \end{itemize}
\end{frame}

\end{document}
